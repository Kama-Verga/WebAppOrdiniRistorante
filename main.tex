\documentclass{article}

% Language setting
\usepackage[italian]{babel}

% Set page size and margins
\usepackage[a4paper,top=2cm,bottom=2cm,left=3cm,right=3cm,marginparwidth=1.75cm]{geometry}

\usepackage{fancyhdr} % Pacchetto per personalizzare intestazioni e piè di pagina
\pagestyle{fancy}     % Abilita lo stile personalizzato
\fancyhead{}
\renewcommand{\headrulewidth}{0pt}
\fancyhead[C]{
    \begin{tabular}{|c|c|}
    \hline
       Web App di Gestione ordini ristorante & Fanesi -- Vergari \\\hline
       Corso Applicazioni Web, Mobile E Cloud  & Data:  26/01/2026\\\hline
    \end{tabular}\newline
}

\fancyfoot[L]{Vergari Andrea m.120200} % Nota nel piè di pagina a sinistra
\fancyfoot[C]{Fanesi Filippo m.124132}
\fancyfoot[R]{\thepage} % Numero della pagina a destra
\renewcommand{\footrulewidth}{0.4pt}

% Useful packages
\usepackage{amsmath}
\usepackage{graphicx}
\usepackage[colorlinks=true, allcolors=blue]{hyperref}

\title{Relazione Progetto Web App di Gestione ordini ristorante}
\author{Vergari Andrea m.120200\\Fanesi Filippo m.124132}

\begin{document}
\maketitle

\renewcommand\contentsname{Indice}
\tableofcontents

\newpage

\section{Obiettivi del progetto}
L’obiettivo del progetto è realizzare una Web App completa (FrontEnd + BackEnd) in grado di gestire gli ordini in delivery di un ristorante.
L’utente può consultare il menù, registrarsi, effettuare il login e creare nuovi ordini inserendo:
\begin{itemize}
    \item un indirizzo di consegna;
    \item una lista di prodotti da ordinare.
\end{itemize}
Il FrontEnd include anche una pagina Orders con filtro per data.

L'applicativo implementa inoltre una regola di business centrale: la gestione di uno sconto legato al ``menù completo''.
Un menù completo è definito come un ordine che contiene almeno (primo, secondo, contorno, dolce)
In presenza di questa combinazione, viene applicato uno sconto fisso del 10\% solamente sui prodotti che appartengono al pacchetto menù.

\subsection{I Requisiti funzionali principali}
\begin{itemize}
    \item Registrazione di un nuovo utente (anche non autenticato).
    \item Login Ed relativa emissione del token JWT.
    \item Visualizzazione del menù (pubblica).
    \item Creazione di un ordine (protetta da JWT).
    \item Visualizzazione ordini (protetta da JWT), con differenza di permessi tra utente standard e admin.
    \item Visualizzazione del singolo ordine ed il suo contenuto.
\end{itemize}

\subsection{Requisiti non funzionali}
\begin{itemize}
    \item Sicurezza: accesso agli endpoint sensibili tramite autenticazione JWT.
    \item Separazione delle responsabilità: codice organizzato per layer per mantenere chiarezza e manutenibilità.
    \item Consistenza dei dati: salvataggio su database relazionale SQL Server in ambiente Azure.
\end{itemize}

\newpage

\iffalse \subsection{Problematiche riscontrate}
In questa sezione vengono riportate alcune problematiche tipiche emerse durante lo sviluppo e le relative soluzioni adottate.

\subsubsection{Problema 1: inserimento manuale su colonna Identity}
Durante i test sul database si è presentato un errore legato all’inserimento esplicito dell’ID in una tabella con colonna Identity (errore SQL Server: inserimento esplicito non consentito con \texttt{IDENTITY\_INSERT OFF}).
\paragraph{Soluzione}
Si è evitato di includere la colonna Identity nella \texttt{INSERT}. In caso di import/migrazioni controllate, è possibile abilitare temporaneamente \texttt{IDENTITY\_INSERT ON}, ma nel sistema standard l’ID viene sempre gestito automaticamente dal DB.

\subsubsection{Problema 2: autorizzazione JWT non applicata correttamente}
In fase iniziale alcune rotte risultavano accessibili anche senza token, a causa di configurazione incompleta dei controlli di autorizzazione.
\paragraph{Soluzione}
E' stato reso obbligatorio l'invio del token JWT sugli endpoint sensibili (creazione ordine e visualizzazione ordini). Lato FrontEnd e stato introdotto un interceptor JWT che legge il token da \texttt{localStorage} e lo allega alle richieste protette.

\subsubsection{Problema 3: calcolo sconto e casi limite}
È stato necessario gestire casi limite come:
\begin{itemize}
    \item ordine con più prodotti della stessa categoria (es. due primi);
    \item ordine che contiene quasi tutte le categorie (manca un solo elemento).
\end{itemize}
\paragraph{Soluzione}
Si è definita una logica deterministica: lo sconto si applica solo se tutte le categorie richieste sono presenti; vengono scontati esclusivamente gli elementi selezionati come pacchetto menù, lasciando invariati gli altri prodotti.

\subsubsection{Problema 4: filtro ordini per data e per utente}
Filtrare per intervallo temporale e, nel caso admin, anche per utente, può portare ad ambiguità sui parametri (es. date uguali, intervallo vuoto, id opzionale non valido).
\paragraph{Soluzione}
Sono state introdotte validazioni sui parametri in input (date obbligatorie e coerenti) e la logica di filtro è stata centralizzata nel layer applicativo, evitando duplicazioni nei controller.

\fi

\newpage


\section{Use-Case}
In questa sezione vengono descritti i principali casi d’uso. Un diagramma UML dedicato sarà aggiunto successivamente.

\subsection{Attori}
\begin{itemize}
    \item \textbf{Utente non autenticato}: può registrarsi, effettuare login e visualizzare il menù.
    \item \textbf{User (ruolo 1)}: può creare ordini e visualizzare la propria cronologia ordini.
    \item \textbf{Admin (ruolo 0)}: può visualizzare la cronologia ordini di tutti gli utenti e filtrare per UserId.
\end{itemize}

\subsection{Casi d’uso principali}
\begin{itemize}
    \item \textbf{UC1 - Registrazione}: l’utente inserisce dati anagrafici e password, il sistema crea un record \texttt{Utente}.
    \item \textbf{UC2 - Login}: l’utente inserisce credenziali, il sistema restituisce un token JWT valido.
    \item \textbf{UC3 - Visualizza Menù}: l’utente consulta i prodotti disponibili.
    \item \textbf{UC4 - Crea Ordine}: l’utente autenticato invia indirizzo e lista prodotti; il sistema calcola totale e salva \texttt{Ordine} e \texttt{ProdottoInOrdine}.
    \item \textbf{UC5 - Visualizza Ordini}: l’utente autenticato richiede la lista ordini (solo propri se User; tutti o filtrati se Admin).
    \item \textbf{UC6 - Visualizza singolo Ordine}: l’utente autenticato richiede tramite ID i dettagli di un ordine.
\end{itemize}

\begin{center}
    \includegraphics[width=12cm]{Diagramma senza titolo.png}
\end{center}

\newpage


\section{Struttura del codice}
\subsection{Struttura BackEnd}
Il BackEnd è organizzato secondo un’architettura a layer (Layered Architecture), con separazione delle responsabilità.
Lo scopo è mantenere il codice più leggibile, testabile e facilmente estendibile nel tempo.

\subsubsection{Organizzazione a progetti}
La soluzione è suddivisa in più progetti:
\begin{itemize}
    \item \textbf{GestioneOrdiniRistorante.Web}: espone le API (controller), gestisce configurazioni e pipeline HTTP.
    \item \textbf{GestioneOrdiniRistorante.Application}: contiene la logica applicativa (service, validator, ecc.).
    \item \textbf{GestioneOrdiniRistorante.Models}: contiene modelli di dominio ed oggetti di trasporto dati (DTO), request e response.
    \item \textbf{GestioneOrdiniRistorante.Infrastructure}: gestisce l’accesso ai dati e l’integrazione con il database.
\end{itemize}

\subsubsection{Database (entità principali)}
Nel database sono presenti le seguenti tabelle principali:
\begin{itemize}
    \item \textbf{Utente}
    \item \textbf{Ordine}
    \item \textbf{Prodotto}
    \item \textbf{ProdottoInOrdine} (tabella di collegamento tra ordini e prodotti ordinati)
\end{itemize}


\subsubsection{Nota sulle ``migrations''}
Le migrations sono un meccanismo (tipico degli ORM) che permette di versionare lo schema del database e aggiornarlo automaticamente.
Nel progetto non risultano adottate: la gestione del database è stata svolta manualmente tramite SQL Server Management Studio.

\subsection{Struttura FrontEnd}
Il FrontEnd Angular e organizzato per funzionalita, con routing per le pagine principali.
\begin{itemize}
    \item \textbf{features}: pagine Login, Register, Menu e Orders (componenti standalone).
    \item \textbf{core}: servizi API (auth, menu, ordini), guard di autenticazione e interceptor JWT.
    \item \textbf{shared}: utility comuni per errori e debug.
\end{itemize}
L'interfaccia usa Angular Material e Reactive Forms per i form e i filtri.

\newpage

\section{Endpoint e API disponibili}
Di seguito sono riportati gli endpoint implementati nel sistema (come da specifica di progetto).

\subsection{Creazione utente}
\textbf{POST} \\
\texttt{https://localhost:7032/Utente/Crea-Utente} \\
Endpoint pubblico (anche non autenticati) per creare un utente.
\begin{verbatim}
{
  "email": "string",
  "nome": "string",
  "cognome": "string",
  "password": "string",
  "ruolo": 0
}
\end{verbatim}

\subsection{Login e token JWT}
\textbf{POST} \\
\texttt{https://localhost:7032/api/v1/Token/Create Token} \\
Endpoint usato per effettuare il login e ricevere un token JWT.
\begin{verbatim}
{
  "email": "string",
  "password": "string"
}
\end{verbatim}

\subsection{Visualizzazione menù}
\textbf{GET} \\
\texttt{https://localhost:7032/Menù} \\
Endpoint pubblico per consultare il menù.

\newpage

\subsection{Creazione ordine}
\textbf{POST} \\
\texttt{https://localhost:7032/Oridine/CreaOrdine} \\
Endpoint protetto da JWT. Inserisce indirizzo di consegna e lista ID prodotti.
\begin{verbatim}
{
  "indirizzo_Di_Consegna": "string",
  "contenuto": [ 0 ]
}
\end{verbatim}
\texttt{contenuto} è una lista di interi che rappresentano gli ID dei prodotti da ordinare.

\subsection{Visualizzazione ordini (con filtri)}
\textbf{POST} \\
\texttt{https://localhost:7032/Oridine/Visualizza Ordini} \\
Endpoint protetto da JWT. Mostra la cronologia ordini:
\begin{itemize}
    \item per l’utente autenticato (User);
    \item per tutti gli utenti (Admin);
    \item filtrata per singolo utente tramite \texttt{idUtente\_Opsionale} (Admin).
\end{itemize}
\begin{verbatim}
{
  "giornoInizio": "2026-01-06T18:23:14.430Z",
  "giornoFine": "2026-01-06T18:23:14.430Z",
  "idUtente_Opsionale": 0
}
\end{verbatim}
Nota: se \texttt{idUtente\_Opsionale} è 0 e il ruolo di chi fa la richiesta è 0 (Admin), vengono visualizzati gli ordini di tutti gli utenti.

\subsection{Visualizzazione ordine}
\textbf{POST} \\
\texttt{https://localhost:7032/Oridine/Visualizza Ordine} \\
Endpoint protetto da JWT. Mostra un ordine da un ID:
\begin{itemize}
    \item per l’utente autenticato (User);
    \item trova l'ordine tramite un ID.
\end{itemize}
\begin{verbatim}
{
  "Id_Ordine": 0
}
\end{verbatim}

\newpage

\section{Sunto del sistema totale}
Il sistema completo funziona end-to-end nel seguente modo:
\begin{enumerate}
    \item L'utente apre il FrontEnd Angular: la pagina Menu e pubblica e mostra i prodotti disponibili.
    \item Registrazione e login salvano il token JWT in \texttt{localStorage}.
    \item L'interceptor JWT allega il token alle richieste protette; la pagina Orders e protetta da guard.
    \item Dal Menu l'utente seleziona i piatti, inserisce l'indirizzo e invia l'ordine.
    \item In creazione ordine, il BackEnd calcola il totale e applica lo sconto del 10\% se l'ordine contiene un menu completo.
    \item Gli ordini sono salvati su SQL Server (Azure); la pagina Orders mostra la cronologia con filtro per data, mentre lato backend restano i permessi User/Admin.
\end{enumerate}

\section{Tecnologie adottate}
In base all’implementazione descritta, le principali tecnologie utilizzate sono:
\begin{itemize}
    \item BackEnd: ASP.NET Core Web API (C\#)
    \item Autenticazione: JWT (JSON Web Token)
    \item Database: Microsoft SQL Server su Azure
    \item Gestione DB: SQL Server Management Studio (SSMS)
    \item FrontEnd: Angular (TypeScript) con componenti standalone e Angular Material
    \item Documentazione e test API: Swagger / OpenAPI (tipicamente integrato nelle Web API)
    \item Versionamento e CI (dalla soluzione): GitHub / GitHub Actions
\end{itemize}


\end{document}
